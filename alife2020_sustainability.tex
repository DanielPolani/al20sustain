\documentclass[letterpaper]{article}

\usepackage{natbib,alifeconf}  %% The order is important

\newcommand{\vectorsym}[1]{\ensuremath{\mathbf{#1}}}
\newcommand{\xextremum}{\ensuremath{x_{\mathrm{ext}}}}
\newcommand{\ccrit}{\ensuremath{c_{\mathrm{crit}}}}
\newcommand{\agentimpact}{\ensuremath{e}}


% *****************
%  Requirements:
% *****************
%
% - All pages sized consistently at 8.5 x 11 inches (US letter size).
% - PDF length <= 8 pages for full papers, <=2 pages for extended
%    abstracts (not including citations).
% - Abstract length <= 250 words.
% - No visible crop marks.
% - Images at no greater than 300 dpi, scaled at 100%.
% - Embedded open type fonts only.
% - All layers flattened.
% - No attachments.
% - All desired links active in the files.

% Note that the PDF file must not exceed 5 MB if it is to be indexed
% by Google Scholar. Additional information about Google Scholar
% can be found here:
% http://www.google.com/intl/en/scholar/inclusion.html.


% If your system does not generate letter format documents by default,
% you can use the following workflow:
% latex example
% bibtex example
% latex example ; latex example
% dvips -o example.ps -t letterSize example.dvi
% ps2pdf example.ps example.pdf


% For pdflatex users:
% The alifeconf style file loads the "graphicx" package, and
% this may lead some users of pdflatex to experience problems.
% These can be fixed by editing the alifeconf.sty file to specify:
% \usepackage[pdftex]{graphicx}
%   instead of
% \usepackage{graphicx}.
% The PDF output generated by pdflatex should match the required
% specifications and obviously the dvips and ps2pdf steps become
% unnecessary.


% Note:  Some laser printers have a serious problem printing TeX
% output. The use of ps type I fonts should avoid this problem.


\title{Quantifying Sustainability in a System of Coupled Tipping Elements}
\author{Jan T Kim, Daniel Polani \\
\mbox{}\\
Adaptive Systems Research Group, School of Engineering and
Computer Science, \\
University of Hertfordshire, Hatfield AL10 9AB,
United Kingdom \\
j.t.kim@herts.ac.uk} % email of corresponding author

% For several authors from the same institution use the same number to
% refer to one address.
%
% If the names do not fit well on one line use
%         Author 1, Author 2 ... \\ {\Large\bf Author n} ...\\ ...
%
% If the title and author information do not fit in the area
% allocated, place \setlength\titlebox{<new height>} after the
% \documentclass line where <new height> is 2.25in



\begin{document}
\maketitle

\begin{abstract}
Systems of coupled elements, with dynamics governed by cubic
differential equations, have been established as simple and powerful
models to study sustainability. Here we combine such systems with an
agent which impacts upon the system. This enables characterising
sustainability as empowerment, a property emerging from the
interaction between the agent and the ecological systems sustaining
it. We present initial results illustrating the effects of dynamical
properties of the system affect on empowerment, and thereby
potentially on sustainability.
\end{abstract}


\section{Introduction}

Sustainability is widely recognised as a central concept that should
underpin decisions and policies, aiming to ensure continued survival,
welfare and prosperity of humankind. In some contrast to its perceived
importance, a rigorous and generally applicable definition of
sustainability continues to prove elusive. Some approaches to define
sustainability, such as the Brundtland definition
\citep{Brundlandcommission1987}, are frequently invoked in the context
of considering Earth as it is, but have not been formalised to enable
their application to other systems, such as Artificial Life models.
More formal approaches to sustainability often focus on stability.
While stability of the sustaining environment (e.g.\ the planet with
its ecological and abiotic systems) is necessary for
sustainability, focusing on it alone excludes the active role of the
entity seeking to be sustained (e.g.\ humankind).

We have previously adapted \emph{empowerment}, an
information-theoretic quantity measuring the future potentialities
that an agent can invoke to characterise sustainability for an agent
inside a given environment \citep{Kim2009_sustainability}. In that
work, empowerment was constrained to only include potentialities when
their effect could be reversed, as a measure for active reversibility
of the agent's environmental exploits. We now modify this idea. We
adapt the concept of tipping elements, introduced in
\citep{Lenton2008_tippingelements}. These are models for ``tipping''
dynamics where a desired locally stable state can be perturbed strongly
enough to move into an undesired stable state.  Bi-stable differential
equations have been used to formally model tipping elements, and these
have been used as building blocks to model and study complex systems
comprised of multiple interacting tipping elements
\citep{Brummitt2015_coupledcatastrophes,Klose2019_interactingtippingelements}.
We show how to combine the empowerment with the tipping element
formalism to offer a framework to model how well an agent can drive a
system ``gone bad'' into its desired state.

% explain that cubic equation systems have been used to explore
% sustainability, but so far the agent to be sustained has not been
% expressly included in such approaches.

% related to May1972: how to determine matrix A? Related to, but not
% identical to interaction matrix in cubic system --> distribution of
% values and calculation of May's \alpha may not be entirely
% straightforward. Note that A depends on current state (local) while
% cubic interaction terms are global.

% Ideas: Demo variant of Brummitt et.al.'s hopping and discuss
% sweep interaction strength and empowerment, look for relation to
% May1972 etc.

\subsection{Additional Introductory Notes}

Cubic differential equations are a simple (arguably the simplest)
systems for studying tipping processes. There always is at least one
stable fixed point, i.e.\ one root of the differential equation where
the first derivative is negative, and if the effective intercept is
within a suitable range, two stable fixed points exist, thus enabling
the system to be in two alternative states.

Sustainability has been characterised based on ecosystems properties,
such as existence of fixed points or robustness to perturbations.
While it is clear (or at least entirely plausible) that some form of
stability or robustness is a necessary condition for sustainability,
this condition is not likely sufficient. System states that are very
stable but would not be considered sustainable are easy to construct.
As an example, increasing the average temperature on earth by more
than 5 degrees might well be considered unsustainable even if it could
be shown that the state of the earth's climate system would be very
stable after that amount of global warming. Typical sustainability
concerns put forward in such a scenario include loss of arable land
caused by e.g.\ rising sea levels and desertification. This reveals
that sustainability includes an aspect of providing ``ecosystem
services'' or generating ``natural capital''.

In the past we have introduced the approach of expressly including the
entity, or the agent, which is to be sustained, in a model
\citep{Kim2009_sustainability}. Here, we demonstrate application of
this approach to systems of coupled tipping elements, which have been
established as a simple formalisation of systems for investigating
aspects of stability and sustainability.



\section{System}

The design of the environment component in our system generally
follows that of \citet{Klose2019_interactingtippingelements}. It consists of
tipping elements, each of which is modelled by a cubic differential equation of the
form
\begin{equation}
  \label{eq_cubic}
  \frac{dx}{dt} = ax - bx^3 + c
\end{equation}
As \citet{Klose2019_interactingtippingelements}, we restrict
ourselves to $a = 1$ and $b = 1$.

The state of a system of $n$ interacting tipping elements is a vector
$\vectorsym{x} = [x_1, \ldots, x_n]^T$ whose interaction
is  modelled by
 a coupling function $C_i(\vectorsym{x})$
\begin{equation}
  \label{eq_coupled}
  \frac{dx_i}{dt} = ax_i - bx_i^3 + c_i + C_i(\vectorsym{x})\;.
\end{equation}
Following \citet{Klose2019_interactingtippingelements}, we use coupling
functions
\begin{equation}
  \label{eq_couplingfunction}
  C_i(\vectorsym{x}) = \sum_j d_{ji} x_j
\end{equation}
but unlike their work, we require $d_{ji} = 0$ if $j \ge i$,
i.e.\ $(d_{ij})$ to be a lower triangular matrix. This restriction
ensures that, for coupled tipping elements, their bi-stable character
is preserved (bi-directional interactions between tipping elements can
result in global dynamical features that amount to a qualitative
departure from the ``tipping point'' system characteristics, such as
cyclic attractors). Thus the graph of interactions becomes a directed
acyclic graph, in which we can arrange the components in order, so
that the steady state of component $x_i$ depends only on higher-ranked
components $x_1, \ldots x_{i-1}$, enabling efficient computation of
fixed points of the system. This models an interrelated, but
hierarchical ``ecosystem'' with different levels of importance for the
subsystems.  Model now the additional effect of the agent
(``humanity'') on its environment by the totality of the terms
$\agentimpact_i(t)$:
\begin{equation}
  \label{eq_coupledwithagent}
  \frac{dx_i}{dt} = ax_i - bx_i^3 + c_i + C_i(\vectorsym{x}) + \agentimpact_i(t)
\end{equation}
This affects the rate of change in the $x_i$; examples would be modelling
emissions of carbon dioxide, release of nutrients
(e.g.\ phosphates), or removal of biomass (harvesting).



\section{Combined Framework}
We present now the core framework.
An individual ``cubic tipping element'' as defined by
Eq.~\ref{eq_cubic} has one or two stable fixed points. In the simple
special case of $c = 0$, there are two stable fixed points which can
be straightforwardly found as $x_{-} = -1$ and $x_{+} = 1$,
\citet{Klose2019_interactingtippingelements} refer to these as the
``normal'' and the ``tipped'' state, respectively. The third, unstable
fixed point at $x_{\mathrm{unstable}} = 0$ is not important for the
work presented here.

Extrema of Eq.~\ref{eq_cubic} are located at
$\pm \xextremum = \pm 1 / \sqrt{3}$. Again considering $c = 0$, the
rate of change at these extrema evaluates to
$\pm d\xextremum / dt = \pm \ccrit = \pm 2 / 3\sqrt{3} \approx \pm
0.38$.
The condition for an individual tipping element to have two stable
fixed points can thus be stated more precisely as
$-\ccrit < c < \ccrit$. This condition generalises to
$-\ccrit < c + C_i(\vectorsym{x}) + \agentimpact_i < \ccrit$ for a
tipping element $i$ in a system of coupled elements subject to impacts
by the agent.

To compute empowerment in the tipping scenario, we determine the
potential stable points that can be ``intentionally'' induced by the
agent. We discretise the state of the environment by distinguishing
only the two stable states $x_{i+}$ and $x_{i-}$ for each element $i$,
i.e.\ a total of $2^n$ states of interest. As in
\citep{Kim2009_sustainability}, we assume no noise in the agent
sensing and actuation. Empowerment is thus determined by the number of
states which the agent can reach by applying its actuators
$\agentimpact_i$.

We use stable fixed points of the environment on its own, i.e.\ of
Eq.~\ref{eq_coupled} as starting points for which we characterise
empowerment, and allow the agent to impact the environment by choosing
to set $\agentimpact_i$ to $-E$, $0$ or $E$, where $E$ is a parameter
setting the strength of the impact, or control, of the agent on the
environment.

If these controls were applied permanently and sufficiently large,
e.g. $E \gg \ccrit + \max_i C_i$, the agent could directly dial the
state of each component, and consequently its empowerment would take
the maximal value of $n$ bits. At the other extreme, setting $E = 0$
would prevent the agent from having any impact on the environment and
its empowerment would vanish in all states.

A crucial concern in sustainability, and about tipping points
specifically, is that human impacts could result in changes to the
environment that are permanent, i.e.\ that stopping an impact may not
be sufficient to restore the environment to its ``normal'' state.
Therefore, we only want to count state changes that persist when the
direct control of the agent is removed. We implement this by letting
the agent apply its impact for a limited time only, which is followed
by a relaxation period during which the environment evolves on its own
and converges towards one of its fixed points. We use the state found
at the end of this relaxation period to quantify empowerment.

In previous work, e.g.\ \cite{Salge2014_empowermentintro} the agent's
reach is typically restricted by a time horizon for its actions. Here,
instead, we will explore empowerment as a function of limits on
$\agentimpact_i$, i.e.\ the strength of effect that the agent can
exert on tipping elements $i \in \{1, \ldots, n\}$.


\section{Implementation}

The software for the analyses presented here was written in R 3.4.4
\citep{RManual2018} and requires the \texttt{deSolve} package for
solving Eq.~\ref{eq_coupled} and
Eq.~\ref{eq_coupledwithagent} using Runge-Kutta 4th order integration.
The code is available at
\texttt{https://github.com/jttkim/al20sustain}.


\section{Results}

\begin{figure}
  \begin{center}
    \includegraphics[width=6cm]{empprof05s.eps}
    % \includegraphics[width=6cm]{empprof04m.eps}
    \includegraphics[width=6cm]{empprof05l.eps}
    \caption{Empowerment as a function of $E$, the impact that the
      agent can exert on element $i$, for the state with most elements
      at $x_{i-}$. Results are shown for a system with weaker coupling
      $d_{ji} \in [-0.2, 0.2]$ (upper panel) and one with stronger
      coupling $d_{ji} \in [-0.8, 0.8]$ (lower panel).}
    \label{fig_empowermentprofiles}
  \end{center}
\end{figure}
We generated the variable parameters $c_i$ and the lower triangle of
$(d_{ji})$ by drawing from uniform distributions. The values of $c_i$
were drawn from $[-0.1, 1]$, and we created an environment with weaker
coupling constants $d_{ji} \in [-0.2, 0.2]$ and one with stronger
coupling constants $d_{ji} \in [-0.8, 2=0.8]$.

Fig.~\ref{fig_empowermentprofiles} shows profiles of empowerment as a
function of $E$, the agent's potential impact on the environment's
tipping elements. At $E = 0$, empowerment is $0$ as the agent is
unable to make any change to the stable state of the environment.
Fig~\ref{fig_empowermentprofiles} shows results for the ``normal''
stable state in which most elements are at $x_{i-}$. Given to the
symmetric method of generating our system's parameters, this
``normal'' is representative for the states of the system in general.

At $E = 0$ the agent has no empowerment, as explained above, and this
extends into small nonzero values of $E$. This is because for an
element in the ``normal'' stable state $x_{i-}$, it is necessary that
$E > \ccrit + C_i$ for the agent to put the element into the
``tipped'' state $x_{i+}$. This explains the emergence of nonzero
empowerment around $E = \ccrit$. Such shifts are somewhat easier with
weaker coupling, which is reflected by the onset of empowerment at
slightly lower $E$.

It is also noticeable from Fig.~\ref{fig_empowermentprofiles} that
weaker coupling enables the agent to attain higher empowerment. One
reason for this is that the number of stable states is maximal in
systems with no coupling at all. In this special case, each element
can be set to each of its two states, independently of all other
elements. At the other end of the scale, a strong interaction from an
upstream element $j$, say with $d_{ji} x_j > \ccrit$, increases the
chance that element $i$ has only one stable state altogether. As a
result, the number of stable system states decreases as the range of
randomly generated $d_{ji}$ increases, explaining also the lower
plateau in empowerment seen with strong coupling.


\section{Discussion and Outlook}

% This deficiency has long been an impediment to public discourse
% \citep{Kim2001_plantbiodiversity}.

We have introduced systems comprised of coupled tipping elements
affected by an agent as a model to study sustainability and to
quantify it using the principled concept of empowerment
\citep{Salge2014_empowermentintro}, rooted in information theory
\citep{CoverThomas1991_informationtheory}. The cubic differential
equation systems, which we use here, capture essential aspects of
sustainability have been applied in sustainability research and
beyond, as reviewed e.g.\ by
\citet{Klose2019_interactingtippingelements}. By including an agent,
our system opens up opportunities to model sustainability as a
property that emerges as an agent interacts with a system. This allows
for a more precise characterisation and formalisation of
sustainability, separating it from more general systems stability,
such as ecosystem stability.

Our plans to further develop this approach include studying the
effects of providing the agent with different actuation channels. An
alternative to allowing the agent to impact all elements equally is to
restrict the agent to impact only one system directly. In this
scenario, high levels of empowerment result if the agent is able to
affect a large number of elements indirectly via their coupling. We
therefore anticipate that stronger coupling will increase, rather than
decrease, empowerment. It will be interesting to investigate whether
susceptibility to ``domino effects'' between elements can be
characterised in this way, and specifically systems in which an
element can cause a downstream element to tip, as described by
\citet{Brummitt2015_coupledcatastrophes}.

Empowerment in its general form considers an agent to receive signals
from the environment, and to send signals to the environment through
its actuators. Both sensors and actuators can be affected by noise. As
humankind, we have only partial and noisy data from our environment,
and our control over our action at a population or society level is
imperfect as well. Therefore, it is important to understand the impact
of these imperfections on sustainability.


% TODO discuss asymmetry
% Shifting the offset c in a tipping unit permits producing a
% hysteresis which makes it easier (for the agent) to "tip" the system
% into the other stable state, while the reverse operation becomes
% more difficult.

It is interesting to note that the offset $c$ produces a hysteresis
which makes it easier(for the agent to tip the system in one direction
(from $x_{-}$ to $x_{+}$ if $c > 0$) while the reverse operation
becomes more difficult. This enables applying the concept of
sustainable empowerment, which we introduced in
\citet{Kim2009_sustainability}, to systems of coupled tipping elements
as well.

Finally, a strength of our model is that it allows efficient
computation of fixed point as well as integrating differential
equations to produce time series. This opens perspectives to
investigate the potential of investigating the use of time series for
characterising and quantifying sustainability, e.g.\ as a tool to
identify risks of breakdowns in sustainability or to identify elements
that are especially relevant to maintaining sustainability.


\subsection{Additional Discussion Items}

The stability of systems of coupled differential equations has been an
area of interest since decades
\citep{May1972_stablelargecomplexsystem,Landi2018_ecologicalnetworks}.

\citet{May1972_stablelargecomplexsystem} considers a very general
class of dynamic systems, characterised locally by a Jacobian. In this
context, stability is defined by fixed points, whereas other
attractors (cyclic or strange) are considered to represent
instability.

We deliberately restrict coupling to a directed acyclic graph. As a
result, all attractors are fixed points, so in the sense of
\citet{May1972_stablelargecomplexsystem} these systems are always
``stable''.

If elements governed by cubic differential equations are allowed to
mutually impact each other (i.e.\ element $i$ impacts on element $j$
and vice versa), it is rather easy to construct systems in which the
elements do not have any fixed points, but are jointly participating
in a cyclic attractor. As a result, these elements can no longer be
characterised as being in one of two discretisable states
(characterised by fixed points labelled ``sustainable'' and
``tipped'', respectively. From this perspective, coupling of cubic
differential equations with a DAG, as used in this paper, can be
considered to be the most general way of coupling that preserves the
``tipping element'' characteristic of the individual elements.



% From perspective for the work presented here include 

%  as minimalistic ALife systems
% to model essential aspects of sustainability in the context of an
% agent trying to maintain and manipulate an ecosystem. We have shown
% the essential properties \ref{hinz und kunz}, and demonstrated that
% this \ref{hurzt und schnurzt}, showing that the system can serve as a
% proof-of-concept framework to study the effect of reversible,
% sustainable management by an agent vs.\ its opposite.

%% However, they provide no inherent way of distinguishing between
%% sustainability and stability. In this paper we use them as a model of
%% an environment which interacts with an agent. This allows us to
%% characterise states that can be reached (and stabilised) by the
%% agent's actuators as candidates for sustainable states.

% perspectives on introducing noise?
% discuss interpretative scenarios, e.g. some "actuation" may benefit
% the agent and add to its ability whereas others may be "cost" --
% e.g. emitting more carbon may allow more Haber-Bosch nitrogen
% fixation...? but adding carbon capture may diminish / be a cost...?

% \section{Preliminary Results}

% \cleardoublepage

% \includegraphics[width=6cm]{coupledtippingdemo_x01.eps}

% \vspace{1cm}

% \includegraphics[width=6cm]{coupledtippingdemo_x02.eps}

% \vspace{1cm}

% \includegraphics[width=6cm]{coupledtippingdemo_x03.eps}

% \vspace{1cm}

% \includegraphics[width=6cm]{coupledtippingdemo_x04.eps}

% \vspace{1cm}

% \includegraphics[width=6cm]{coupledtippingdemo_x05.eps}

% \vspace{1cm}

% \includegraphics[width=6cm]{coupledtippingdemo_x06.eps}

% \vspace{1cm}

% \includegraphics[width=6cm]{coupledtippingdemo_x07.eps}

% \vspace{1cm}

% \includegraphics[width=6cm]{coupledtippingdemo_x08.eps}

% \vspace{1cm}

% \includegraphics[width=6cm]{coupledtippingdemo_x09.eps}

% \vspace{1cm}

% \includegraphics[width=6cm]{coupledtippingdemo_x10.eps}



\footnotesize
\bibliographystyle{apalike}
\bibliography{bioinfo}
%\bibliography{bioinfo} % replace by the name of your .bib file
\end{document}


%%% Local Variables:
%%% mode: latex
%%% TeX-master: "ieee_alife_2020_sustainability"
%%% End:
